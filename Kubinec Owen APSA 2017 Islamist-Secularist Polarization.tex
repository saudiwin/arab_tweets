\documentclass[12pt]{article}
\fontfamily{times}
\usepackage{url}
\usepackage[margin=1in]{geometry}
\usepackage{array}
\usepackage{ragged2e}

\usepackage{fullpage}
\usepackage{parskip}
\usepackage{titling}
\usepackage{tikz}
\setlength{\parindent}{15pt}
\usepackage{times}
\usepackage{booktabs}
\usepackage{siunitx}
\usepackage{graphicx}
\usepackage{amsmath}
\usepackage{authblk}
\usepackage[american]{babel}
\usepackage{csquotes}
\usepackage[hidelinks]{hyperref}
\usepackage[authordate,backend=biber]{biblatex-chicago}

\addbibresource[location=remote]{https://virginia.box.com/shared/static/m0nw01c4bcltkauwy2hd1467vu7dufrj.bib}
\usepackage[section]{placeins}

\author{Robert Kubinec and John Owen}
\affil{\small Department of Politics \\
	\small University of Virginia}
\date{\small \today}

\linespread{1.5}

\title{Islamist-Secularist Protest during the Arab Uprising}

\begin{document}

\maketitle

\abstract{Scholars continue to disagree as to how far contentious politics diffuses within and across states and by what mechanisms it does so. We use new data and empirical measures to test polarization during and after the Arab Uprisings of 2010-12. After authoritarian governments began to fall, populations in several states began to polarize between secularists and Islamists over what kind of regime was to replace the ousted one. We hypothesize that this Islamist-secularist polarization was triggered by catalytic events (such as Muslim Brotherhood electoral victory in Egypt) and diffused transnationally owing to social media and satellite television, dividing anti-status-quo actors throughout the region. To examine polarization over time, we collected a comprehensive dataset on elite and citizen Twitter accounts across Arab countries after the Arab Spring. Using item-response models, we model polarization as the difference in latent ideological position between elites, and we show how polarization within countries changes over time in response to exogenous political shocks. By doing so we are the first to offer compelling statistical evidence of the endogenous process of polarization across competing ideological groups and states.\footnote{\linespread{.5} Prepared for presentation at the Annual Meeting of the American Political Science Association, San Francisco, 1 September 2017. Copyright \textcopyright 2017 by the American Political Science Association.\\
		The authors thank the Amb. Henry J. and Mrs. Marion R. Taylor Chair at the University of Virginia for research funding.  John Owen thanks the Liu Institute for Global Inquiry at the University of British Columbia for providing a research venue. Robert Kubinec thanks the University of Virginia's Quantitative Collaborative for computing resources and Jonathan Kropko for helpful feedback on this project.}}








\section*{Introduction}

The Arab Uprising of 2010-13 was an exceedingly complex phenomenon.  The initial event – the self-immolation of Mohamed Bouazizi in Sidi Bouzid, Tunisia – triggered a chain of events marked by interaction and feedback loops.  During the Uprising, experts predicted a particular unfolding or winding down of the phenomenon, and often were wrong.  Some assured their readers that the Tunisian unrest of December 2010 and January 2011 would not spread to other Arab countries (Karon 2011; Walt 2011); others predicted that the unrest that did spread to Libya, Egypt, Yemen, Syria, and elsewhere would bring liberal democracy to those countries (Obama 2011, Juppé 2011 – any non-politicians?).  In other words, two of the unpredictable developments were the spread of unrest across countries and the evident shift in the goals of rebels.  
A helpful approach to explaining, if not predicting, the progress of the Arab Uprising and infectious political unrest more generally is via group polarization, or the endogenous segregation of a population by preferences into two or more mutually antagonistic groups.  

Group polarization is situational and hence may be short-lived; it is different from the long-term, materially based social polarization that many social scientists study (Moulaert et al. 2003).  Group polarization is a way to conceive of how identities and preferences change in response to exogenous events such as a public demonstration or a coup d'état.  A stylized version runs as follows:  At time t, an unmodeled event takes place (e.g., Bouazizi's self-immolation); at t+1, the news spreads; at t+2, a portion of the population of Sidi Bouzid becomes angry and identifies more strongly against the local and national government, as reflected in anti-government speech and action; at t+3, news of that speech and action spreads; at t+4, a portion of Sidi Bouzid's population becomes angry at the protesters and identifies more strongly with the government, as reflected in pro-government speech and action; at t+5, news of that speech and action spread within Tunisia and beyond; and so on.  Group polarization is self-reinforcing.  It may be slowed, halted, or reversed by various developments, including coercion (censorship, physical force) by governments.

Its situational quality, specifically its dependence on exogenous events, makes group polarization impossible to predict, but therefore especially helpful in understanding complex chains of events such as the Arab Uprising.  The group-polarization approach presupposes, with a long tradition in social theory, that people have multiple group affiliations; that belonging to one group entails defining oneself over against one or more alternative groups; that for a given individual at a given time a particular group affiliation may be more or less salient; and that individuals respond to signals of friendship and hostility from one another; and hence that populations sometimes polarize along one axis of identity, temporarily submerging other axes of identity.  Of particular interest in the Arab Uprising are (1) cross-polarization, or the shift from polarization along one axis (e.g., pro- versus anti-regime) to polarization along a cross-cutting axis (e.g., Islamist versus secularist), and (2) transnational polarization, i.e., simultaneous polarization along an identity axis in two countries.

Group polarization is a phenomenon of changing sentiment, identity, and preference.  Measuring it with rigor has been difficult, and thus so has testing claims about what triggers it and what suppresses it.  The growing use of social media, the availability of the resulting data, and advances in computing power, however, combine to allow researchers to measure changes in group polarization over time.  In this paper, we present analysis of Twitter data during 2013 in Egypt and Tunisia.  We choose this time period because Egypt went through a number of political events that seem prime candidates for triggers of group polarization.  Most obvious is the July 3 coup d'état in which secularist military officers overthrew the elected government of Mohamed Morsi of the Muslim Brotherhood.  Other potentially polarizing events include clashes between Muslims and Christians in al-Khousos on April 6-7; Morsi's appointment of seven new governors from the Muslim Brotherhood on June 16; and the military government's violent dispersal of pro-Morsi demonstrators in Cairo on August 14 (Daily News Egypt, https://dailynewsegypt.com/2013/08/14/live-updates-pro-morsi-sit-ins-dispersed/, retrieved on 11 August 2017.) 

This paper presents a Bayesian item response theory model that incorporates feedback loops between ideological groups through the use of co-integrated time-series priors on latent constructs. The item response theory model has been widely used to study social and group polarization, but we are the first to show how to explicitly model feedback by proposing that similar cross-country ideological groups will tend to show auto-correlated trends over time. In other words, we expect that the differences between the ideological positions of secularists in Egypt and secularists in Tunisia will be roughly proportional (i.e., stationary) over time, while the same condition will hold for Islamists in both countries. 

In what follows, we explicate the logic of group polarization in an informal model.  We include propositions about de-polarization and cross-polarization.  We offer hypotheses about group polarization, and explain and defend our use of tweets in Egypt and Tunisia in 2013.  We present the statistical model and data analysis.  We close with thoughts for future research.


\section*{Group Polarization: An Informal Model}
Polarization has been studied extensively by social scientists.  Much of that work concerns social polarization, or segregation into groups that are stable over long periods of time (such as ``Red America" and ``Blue America").   By group polarization, we mean a process of segregation – not an equilibrium – that is relatively short-term or situational yet may be politically consequential.   Along with other scholars, we define polarization as a social construct, namely as progressive identity change that entails preference change.   When two actors polarize, at time $t$, both actors may prefer a 50-50 allocation of goods; at $t+1$, each may prefer a 60-40 allocation in its favor; at $t+2$, a 75-25 allocation; and so on.  At the limit of polarization, each side wishes the other destroyed.   Group polarization, then, is one way to formulate a progressive worsening of conflict; it does not cause conflict, in the sense of an independent variable causing a dependent one.  Rather, polarization is conflict that is self-intensifying.  Group polarization is endogenous, not in the sense that it is unrelated to pre-existing cleavages bur rather in the sense that, once triggered by exogenous events, it is self-intensifying.   It entails the altering of individuals' preferences and practices and creates new threats and opportunities for various actors, including actual and aspiring rulers.  

Stated informally, the basic group polarization model is simple.  Assume a population of 100 persons, all 100 belonging to one half or another of $x$ pairs of opposing social groups (labor or capital, democrats or authoritarians, Islamists or secularists, urban or rural, etc., etc.).  Fifty are pro-democracy, fifty pro-authoritarian.  These groups do not correlate significantly to any other groups; e.g., democrats are as likely to be Islamist as secularist.  The population thus has cross-cutting cleavages.   At time $t$, the population begins in an unbiased equilibrium, such that, although individuals may identify more strongly with one group affiliation than with others, in the population as a whole, no identity axis predominates; hence social interaction does not skew the distribution of resources, including information, to any of the social groups (Dunning and Harrison 2010).  Now suppose that at t+1 three democrats – one Islamist and two secularist, and two urban and one rural – publicly beat an authoritarian.  Assuming a relatively free flow of information, that event can trigger the polarization of the population along a democratic-authoritarian ideological axis, such that democrats and authoritarians care less and less about class, being urban or rural, or mosque-state questions and more and more about ideology.  If not disrupted, polarization by definition culminates in inter-group violence.

Transnational group polarization takes place when polarization spans two or more countries at once.  Citizenship in a state amounts to yet another group affiliation, albeit normally an especially politically salient one that carries the advantages of a state apparatus.  States are set up to foster group identity and loyalty vis-à-vis foreigners.  They may use physical segregation, closed or semi-closed national borders, national economic integration, propaganda, history, threats of war, coercion, and other means to induce a strong national identity among citizens and hence a strong sense that foreigners are an ``other."  Yet, interaction – communication, trade, investment, travel – across state borders is normal, particularly among most countries in the twenty-first century.   States vary in their capacity to build and maintain a national identity and to have that identity perpetually trump all other group affiliations, including transnational ones, across all conditions.   Thus transnational group affiliations – ethnic, religious, ideological, class, sexual – are part of life for people in most countries.  Insofar as communication across state boundaries is uncensored by states, transnational group affiliations can yield transnational group polarization.  The informal model above may then incorporate democrats and authoritarians in a second state (and a third, a fourth, and so on).

Cross-polarization takes place when a group at time $t-1$ is not in unbiased equilibrium, but instead is polarized along one identity axis and at t an event triggers polarization along a different axis.  In the example above, at t the population is segregated according to preference into Islamists and secularists.  At $t+1$, three authoritarians (say, one Islamist and two secularist) publicly beat a democrat.  At $t+2$, as Islamists and secularists who are democrats begin to identify more as democrats and less as Islamists or secularists.  Polarization of the entire population along a democrat-authoritarian axis will commence.

Group de-polarization – understood as a diminution of speech and acts biased according to group affiliation – may set in when communication is censored or degraded or speech and action forcibly curtailed.  The most obvious agent of de-polarization is a government, which typically has at least some of the means of censorship and coercion at its disposal.  A government threatened by polarization into pro- and anti-government groups can be expected to try to slow, halt, or reverse it – or to trigger a cross-polarization that would weaken the anti-government group.  

\section*{Justifying Assumptions}
Social identity theory links the formation of groups and their degree of competition by means of the concept of polarization.  Microfoundations for such a model are in philosophy and social theory.  Assume that persons are not atomized individuals whose fundamental goal is to maximize their own exogenously derived utility and who value the gains and losses of others only insofar as those are instrumental to such maximization.  Assume instead the persons depicted by traditions in sociology (Simmel 1955; Coser 1956):  each individual is fundamentally a member of social groups, and he identifies his interests to some extent with those of the groups to which he belongs and against opposing groups.

The logical foundations of this communal psychology is seen in the formula articulated by Spinoza and, later, Hegel:  omnis determinatio est negatio, or ``all determination is negation" (Melamed 2013).  A thing must necessarily have properties, such as ``short" or ``cold."  But properties only exist in contrast to other properties (Taylor 1975, 232-39).  Human being contrasts to non-human being (animals, plants, rocks); female, to male; labor, to capital; old, to young; and so on.  Having a self requires having an other.   Having a property is equivalent to belonging to the set of things that have that property (Quine 1989, 22-26).  Being female is equivalent to being a member of the set of persons that are female.  Identity thus is social:  who I am entails my group memberships.

Philosophers may disagree on the soundness of this logic, but experimental evidence suggests that people, or at least some people, tend to think, feel, and act according to it.  People tend to be self-interested, but their notion of ``self" may expand to include persons in their social group whose existence requires contrast with some opposing or ``out-group" (Mercer 1995).  Indeed, these two identifications are mutually constitutive.  As Simmel put the matter, ``It appears to be necessary for us human beings, whose whole psychical nature is built upon our sensitiveness of difference, that a feeling of separateness should always exist alongside of the feeling of unity to make this latter perceptible and tangible" (Simmel 1898, 45-46).  

\section*{Overlapping Social Groups and Different Saliencies}
That individuals belong to multiple social groups, each with a corresponding anti-group, introduces a complication.  For Simmel, an individual's identity consists of the unique overlap of the groups to which she belongs (Simmel 1955, 139-41).  Yet, a given individual will identify more strongly with some of his groups than with others.  
Sometimes large groups of individuals do so simultaneously, such that populations polarize along a particular axis of group identity.   Social-psychological literature posits at least two attributes of groups that lend them high salience.  One is prestige or high status:  experiments show that members of high-status groups are significantly more biased toward fellow members and against nonmembers than are members of low-status groups.   Experiments also show that a second attribute is threat (physical, economic, status, etc.) – particularly among persons already highly committed to the group.  A new threat – such as an attack on a group member by members of the opposing group – tends to arouse in such persons fears that they may be next, and so they tend to increase their biases toward that particular group affiliation.  They identify more with it and against the threatening group.  This experimental result is anticipated by Simmel:
It is a fact of the greatest social significance, one of the few which are true almost without exception of group formations of every sort, that common antagonism against a third party under all circumstances tends to consolidate the combining group, and with much greater certainty than community in friendly relationships toward a third party. 

If these attributes of prestige and threat are associated with high salience, it should be the case that a rise in a group's status or endangerment can render it more salient for its members.  A rise in status may be triggered in politics by a victory in an election or a civil war, or an unexpectedly large public rally.  A rise in threat may be brought on by physical violence, verbal abuse, or evidence of discrimination or persecution against the group.

Social-psychological literature notes that people vary by level of commitment to various groups.   In equilibrium some city-dwelling Islamists identify more as urban and less as Islamist; others identify more as Islamist as less as urban.  This kind of heterogeneity could in principle stifle polarization, because low-commitment group members could try to exit or hide from the group rather than take the risks that come with strongly identifying with it.  Against that possibility, Tilly writes that, following a triggering event, highly committed group members mediate and broker polarization by spreading information about the threat or increased status and about ongoing polarization.   Such brokers may propagandize by exaggerating and inventing symbolic events.  Public discourse turns to what is to be done; those who hold extreme views tend to have more influence in such times and moderates either are quiet or move toward the extreme.   Smith (JPET, 2012) models allocation decisions in a game comprising two social groups, each comprising two types of actors:  ``behavioral" actors who are biased to favor their own group members, and ``rational" actors who are unbiased.  The model shows that ``rational" across will come to act like ``behavioral" ones and favor allocation of goods to their own group.  

It stands to reason that the low-commitment actors posited by Ellemers et al. would behave like Smith's ``rational" actors.
In sum, an exogenous event that either raises the prestige of social group $A$ or threatens group $A$ may cause people who belong to multiple overlapping groups $A$ through $Z$ to identify more strongly with $A$ and against $\neg A$ and less with $B$ and against $\neg B$, etc.  Increases in status and in threats may be simultaneous:  an increase in $A$'s status may simultaneously threaten members of $\neg A$ and thus cause them to identify more as $\neg A$s and against group $A$.  Large public anti-government demonstrations, as take place during a typical political spring, can both raise the status of being anti-government and simultaneously threaten those who identify with the government.  And again, polarization tends to feed on itself:  as members of $A$ observe members of $\neg A$ identifying more as $\neg A$s, members of $A$ will identify still more strongly with $A$; and so on. 
 
\section*{Data Collection}
Analyzing these phenomena quantitatively is difficult because we need good measures of ideology across countries that also varies over time. The advent of social media during the Arab Uprisings provides some of the first available data that we can use to examine the predictions of the theory. We chose to use Twitter due to its public nature and the ability to closely track elite users, or those users with a large number of followers. We built a sample of Egyptian and Tunisian twitter users that are broadly representative of political thinking \& discourse within these countries.

To obtain the sample, we started with a universe of tweets from the early stage of the Arab Spring, December 2010 to April 1st, 2011 that all matched the search times ``Cairo", ``Alexandria" and ``Tunis" in the location field in Twitter.  This dataset of 11 million tweets parsed in order to identify users that had received the largest number of retweets and mentions during that time period. This list was then curated, both by removing accounts that were later abandoned and by adding ideological diversity. In general, Twitter in 2011 was dominated by well-educated secular elites with a flair for democracy, while Islamists and pro-regime figures were later to adopt Twitter as a medium. For example, in Egypt we added the account for Mohammed Morsi, the Islamist president affiliated with the Muslim Brotherhood, and in Tunisia we added Rached Ghannouchi, long considered to be the guiding force of the Islamist Nahda movement in Tunisia. The full list of users selected for analysis is available in the appendix.

The final sample amounted to 165 Twitter users, 64 from Tunisia and 101 from Egypt. The larger number from Egypt reflects the much larger Twitter-sphere in the country and hence the need to obtain a broader sample of users. We then had two graduate assistants code the user list along two axes: Islamist-Secularist and Pro/Anti Democracy. The first axis is easier to identify in the data and is the one that we focus on in this paper because the election of Islamists in Egypt and Tunisia's initial plebiscites after democratization caused many elite Twitter users to weigh in, providing strong evidence of their views on politics and religion. In addition to the binary classification, we also had the graduate coders record their confidence in their assessment on a scale of 0 to 100. We further reviewed cases that had an uncertainty of less than 50 percent. In general, these users did not tweet as much on political topics and their ideology is relatively unknown. We excluded these users for the analyses we report here as their lack of ideological content makes them uninteresting to this analysis.

With this finished list of 165 elite users, we then collected their full tweet histories from March 31st, 2013 to December 31st, 2013. This 1.7 million list of tweets was then filtered down to 1.2 million tweets that constitute each user's own composed tweets by removing all of each user's retweets. Using Twitter's open API, the retweets of all of these 1.2 million tweets was then downloaded as a list of user IDs for each user per day for a total of 275 days. The use of the open API represents a limitation in the data collection because only 100 re-tweets of a given retweet may be downloaded; however, this limit was rarely reached in practice because very few of the tweets in question had more than 100 retweets. Even ex-President Mohammed Morsi, who has more than a million twitter followers, averaged only a few hundred re-tweets per tweet during 2013. Nonetheless, this data must still be understood as a sample of the full number of retweets, especially for users with a very large popular following.

The final database comprises a set of elites $j$ and citizens $i$ in which the outcome is the number of times that $i$ retweeted $j$ for each day in the sample. The set of elites is further divided into four groups: Islamists in Egypt and Tunisia versus secularists in Egypt and Tunisia. We can then think of each group having an average position in an ideological ideal point space that varies over the number of days in the sample. 

\section*{Modeling Ideological Polarization Over Time}
To adequately model these phenomena, we combine two distinct literatures in statistics: time-series econometrics, in particular co-integrated equations, with the item-response theory literature on estimating latent concepts. 







\end{document}